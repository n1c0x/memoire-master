\chapter{Analyse de l'existant}

Intro

\section{Partie 1}

Intro

\subsection{Sous-partie 1}

Bla

\subsection{Sous-partie 2}

Bla\\

Transition

\section{Partie 2}

Bla\\

Transition

\section{Bilan récapitulatif}

Voici un tableau (cf. fig. 2.1) récapitulatif de notre analyse de l'existant...\\

%tableau centré à taille variable qui s'ajuste automatiquement suivant la longueur du contenu
\begin{figure}[!h]
\begin{center}
\begin{tabular}{|l|l|l|l|l|}
  \hline
  Solution & Critère 1 & Critère 2 & Critère 3 & Critère 4\\
  \hline
  Solution 1(cf. ref. \cite{cite0}) & Oui & Oui & Oui & Oui \\
  Solution 2(cf. ref. \cite{cite1}) & Oui & Oui & Oui & Non \\
  Solution 3(cf. ref. \cite{cite2}) & Oui (sauf telle chose) & Non & Non & Oui\\
  Solution 4(cf. ref. \cite{cite3}) & Oui& Non & Oui & Non\\
  Solution 5(cf. ref. \cite{cite4}) & Oui (uniquement ceux-ci) & Non & Oui & Non\\
  \hline
\end{tabular}
\end{center}
\caption{Tableau récapitulatif des solutions}
\end{figure}
